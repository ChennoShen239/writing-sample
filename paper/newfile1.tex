%% LyX 2.4.3 created this file.  For more info, see https://www.lyx.org/.
%% Do not edit unless you really know what you are doing.
\documentclass[12pt,english,final]{article}
\usepackage{fourier}
\usepackage[T1]{fontenc}
\usepackage[utf8]{inputenc}
\synctex=-1
\usepackage{babel}
\usepackage{booktabs}
\usepackage{mathtools}
\usepackage{amsmath}
\usepackage{amsthm}
\usepackage{amssymb}
\usepackage{graphicx}
\usepackage[letterpaper]{geometry}
\geometry{verbose,tmargin=1.35in,bmargin=1.35in,lmargin=2cm,rmargin=2cm,footskip=1.5cm}
\usepackage{setspace}
\usepackage[authoryear]{natbib}
\setstretch{1.3}
\usepackage[]
 {hyperref}

\makeatletter

%%%%%%%%%%%%%%%%%%%%%%%%%%%%%% LyX specific LaTeX commands.
%% Because html converters don't know tabularnewline
\providecommand{\tabularnewline}{\\}
%% A simple dot to overcome graphicx limitations
\newcommand{\lyxdot}{.}


%%%%%%%%%%%%%%%%%%%%%%%%%%%%%% Textclass specific LaTeX commands.
\theoremstyle{definition}
\newtheorem{defn}{\protect\definitionname}
\theoremstyle{plain}
\newtheorem{prop}{\protect\propositionname}
\theoremstyle{plain}
\newtheorem{thm}{\protect\theoremname}
\theoremstyle{plain}
\newtheorem{cor}{\protect\corollaryname}

%%%%%%%%%%%%%%%%%%%%%%%%%%%%%% User specified LaTeX commands.


\usepackage{babel}
\usepackage{subcaption}
\usepackage{diagbox}


\usepackage{appendix}


\usepackage{fancyvrb}
%geometry (sets margin) and other useful packages
\usepackage{booktabs}\usepackage{epstopdf}%finish proof here
\usepackage[font=normalsize]{caption}
\usepackage{enumerate}
\usepackage{mathrsfs}
\usepackage{pdfpages}
%\usepackage{setspace}
\usepackage{url}
%\usepackage{lipsum}



\definecolor{darkblue}{rgb}{0,0,.6}
\definecolor{darkred}{rgb}{.6,0,0}



%% Output encoding, for accented characters, etc.



%%
%% Macros and notation changes
%%

% Math notation macros
\newcommand{\diag}{\text{diag}}
\newcommand*{\expt}{\mathbb{E}}
\newcommand*{\Var}{\mathbb{V}}
\newcommand*{\fe}{\mathbb{F}}
\newcommand*{\pr}[1]{\mathbb{P} \left\{ #1 \right\}}
\newcommand{\hs}{\mathcal{H}}
\newcommand{\gs}{\mathcal{G}}
\newcommand{\fs}{\mathcal{F}}
\newcommand{\ft}{\mathcal{F}_\tau}
\newcommand{\id}{\mathbb{I}}
\newcommand{\rn}{\mathbb{R}}
\newcommand{\nn}{\mathbb{N}}
\renewcommand{\vec}[1]{\mathbf{#1}}
\newcommand{\gvec}[1]{\boldsymbol{#1}} % maintains greek letters correctly
%\newcommand{\comment}[1]{{\color{red}#1}} % comments that show up in the PDF

\newcommand{\sgn}{\text{sgn}}
\newcommand{\argmax}{\arg \! \max}
\newcommand{\argmin}{\arg \! \min}

% Underbar does not change font
\def\munderbar#1{\underline{\sbox\tw@{$#1$}\dp\tw@\z@\box\tw@}}



% Switch the "straight epsilon" and the "curly epsilon"
\let\temp\epsilon
\let\epsilon\varepsilon
\let\varepsilon\temp

%%
%% Theorem environments
%%


\renewcommand{\baselinestretch}{1.32}


\newcommand{\commentout}[1]{}


\renewcommand{\paragraph}{\@startsection{paragraph}{4}{\z@}%
  {1.5ex \@plus1ex \@minus.2ex}%
  {-1em}%
  {\normalfont\normalsize\bfseries}}


% ----------------------------------------
% ----------------------------------------

\theoremstyle{plain}
\newtheorem{assumption}{\protect\assumptionname}\providecommand{\assumptionname}{Assumption}


\usepackage{hyperref} % no options
\hypersetup{
  unicode=true,
  bookmarks=false,
  breaklinks=false,
  pdfborder={0 0 1},
  colorlinks,
  linkcolor=darkred,
  citecolor=darkblue,
  urlcolor=darkblue,
  pdftex
}


\usepackage{setspace}     % for line spacing
\usepackage{footmisc}     % for footnote customization
\usepackage{titling} % provides \droptitle

\makeatother

\providecommand{\corollaryname}{Corollary}
\providecommand{\definitionname}{Definition}
\providecommand{\propositionname}{Proposition}
\providecommand{\theoremname}{Theorem}

\begin{document}
\title{Sovereign Default with Bounded Rationality}
\author{Chen Gao\thanks{National School of Development, Peking University. Email: \protect\href{mailto:chengao0716@gmail.com}{chengao0716@gmail.com}}}
\maketitle

\section{Model }

This section outlines the baseline model, which extends the framework
of \citep{10.1257/aer.98.3.690}. The decision maker is a government
of a small open economy that borrows from risk-neutral foreign creditors. 

\paragraph{Output}

A small open economy is endowed with an exogenous stochastically fluctuating
potential output stream $\left\{ y_{t}\right\} _{t=0}^{\infty}$.
The logarithm of potential output, denoted $\ln(y_{t})$, follows
a first-order autoregressive (AR(1)) process:
\begin{equation}
\ln(y_{t+1})=\rho\ln(y_{t})+\varepsilon_{t+1},\quad\text{where }\varepsilon_{t+1}\sim N(0,\sigma_{\varepsilon}^{2})\label{eq:ar1_y}
\end{equation}
Here, $y_{t}$ is the level of potential output, $\rho$ is the persistence
parameter (e.g., $|\rho|<1$), and $\varepsilon_{t+1}$ is an i.i.d.
normal shock with mean zero and variance $\sigma_{\varepsilon}^{2}$.
The process for $\ln(y_{t})$ implies a transition probability kernel
for the level of output, $p(y,y')$. Potential output $y_{t}$ is
realized only in periods in which the government honors its sovereign
debt. The output good can be traded or consumed. Households within
the country are identical and rank stochastic consumption streams
according to 
\[
\mathbb{E}_{0}\left[\sum_{t=0}^{\infty}\beta^{t}u\left(c_{t}\right)\right],
\]
where $u\left(\cdot\right)$ is an increasing and strictly concave
utility function. Household consumption $\left\{ c_{t}\right\} _{t=0}^{\infty}$
is affected by the government’s international borrowing and lending
decisions. Because households are averse to consumption fluctuations,
the government will try to smooth consumption by borrowing from (and
lending to) foreign creditors.

\paragraph{Asset Market }

The only credit instrument available to the government is a one-period
bond traded in international credit markets. The bond market has the
following features: first, the bond matures in one period and is not
state contingent. Second, a purchase of bonds with total face value
$B'$ (a claim to $B'$ units of the consumption good next period)
costs $q(B',y)B'$ today, where $q(B',y)$ is the price of a bond
promising one unit of face value next period, given current state
$y$ and chosen next period assets $B'$. Third, if the government
issues bonds (i.e., borrows, $B'<0$), it sells a promise to repay
$-B'$ units next period (note that $-B'$ is a positive quantity).
For this, it receives $q(B',y)(-B')$ units of the good today.

Earnings on the government portfolio are distributed (or, if negative,
taxed) lump sum to households. When the government is not excluded
from financial markets, the one-period national budget constraint
is 
\begin{equation}
c_{t}=y_{t}+B_{t}-q\left(B_{t+1},y_{t}\right)B_{t+1}.\label{eq:budget_cons}
\end{equation}
To rule out Ponzi schemes, we also require that $B_{t}\ge-Z$ in every
period where $Z$ is an exogenous borrowing upper bound. $Z$ is chosen
to be sufficiently large that the constraint never binds in equilibrium.

\paragraph{Financial Markets }

There are risk-neutral foreign creditors who know the domestic output
stochastic process $\left\{ y_{t}\right\} _{t=0}^{\infty}$and observe
$\left\{ y_{s}\right\} _{s=0}^{t}$ at time $t$. The creditors can
borrow or lend without limit in an international credit market at
a constant risk-free rate $r$. The creditors receive full payment
if the government chooses to pay and receive $0$ if the government
defaults on its one-period debt due. 

If the government is expected to default on obligations due at $t+1$
with probability $\delta(B',y_{t})$ (where $B'$ is the chosen bond
position for $t+1$ and $y_{t}$ is the output at time $t$), the
expected value of a promise to pay one unit of consumption next period
is $1-\delta(B',y_{t})$. Therefore, the bond price $q(B',y)$ for
a promise to pay one unit of consumption next period is given by its
discounted expected value:
\begin{equation}
q\left(B',y\right)=\frac{1-\delta\left(B',y\right)}{1+r}.\label{eq:qdelta}
\end{equation}
where $\delta(B',y)$ is the probability of default next period, conditional
on choosing $B'$ today when current output is $y$. The key is to
determine the default probability $\delta(B',y)$.

\paragraph{Decision on default}

At each time $t$, the government chooses between defaulting and meeting
its current obligations and purchasing or selling an optimal quantity
of one-period sovereign debt. Defaulting means declining to pay all
of its current obligations. Default triggers two consequences: first,
the output is decreased from $y_{t}$ to $h\left(y_{t}\right)$, where
$0\le h\left(y_{t}\right)\le y_{t}$. For instance, $h(y_{t})$ could
take the form $\min(\psi\bar{y},y_{t})$, where $\psi$ is a parameter
and $\bar{y}$ represents a reference output level (such as the long-run
average output).

Output returns to normal only after the sovereign regains access to
international credit markets. Second, the sovereign loses access to
foreign credit markets. While in a state of default, the economy regains
access to foreign credit in each subsequent period with probability
$\theta$.

\subsection{The Government's Problem}

\paragraph{Default}

Let $V^{D}\left(y\right)$ be the value function of government in
default, the government has only one state variable $y_{t}$ and facing
the recursion:
\begin{equation}
V^{D}\left(y\right)=u\left(h\left(y\right)\right)+\beta\mathbb{E}_{y'}\left[\theta V\left(0,y'\right)+\left(1-\theta\right)V^{D}\left(y'\right)|y\right].\label{eq:val_default}
\end{equation}
In the continuation value, the government either regain credit market
access with $0$ debt with probability $\theta$, or stay in default
and gain $V^{D}\left(y'\right)$ next period with probability $1-\theta$.

\paragraph{Repay}

If the government stays in credit market at time $t$ and chooses
to repay, then it choose a control variable $B_{t+1}$ to maximize
the value of repayment. The Bellman equation is 
\begin{equation}
\begin{aligned}V^{R}\left(B,y\right) & =\max_{B'}\left\{ u\left(c\right)+\beta\mathbb{E}_{y'}\left[V\left(B',y'\right)|y\right]\right\} \\
\text{s.t. \ensuremath{\quad}}c & =y+B-q\left(B',y\right)B'\\
B & \ge-Z
\end{aligned}
,\label{eq:val_repay}
\end{equation}
where $V\left(B',y'\right)$ is the optimal value of the government's
problem when facing the choice of whether to repay or default at the
beginning of the next period with state $\left(B',y'\right)$. The
government chooses to default if and only if $V^{D}\left(y\right)>V^{R}\left(B,y\right)$
and thus $V\left(B,y\right)$ is given by the maximum of the two values:
\begin{equation}
V\left(B,y\right)=\max\left\{ V^{R}\left(B,y\right),V^{D}\left(y\right)\right\} .\label{eq:val_all}
\end{equation}


\paragraph{Default probability with $\lambda$- rationality}

Given the value function, global lenders can now evaluate the probability
of default and therefore give the price function $q\left(B',y\right)$.
Unlike \citet{10.1257/aer.98.3.690} where all lenders are rational
in comparing the next period value, I introduce a parameter $\lambda\in\left[0,1\right]$
to denote the fraction of ``rational'' lenders. For the rational
lenders, they predict the default decision of all possible next period
output level $y'$ and then formulate expectation of default next
period given $B'$ as:
\begin{equation}
\begin{aligned}\delta_{r}\left(B',y\right) & =\mathbb{E}_{y'}\left[\mathbb{I}_{\left\{ V^{D}\left(y'\right)>V^{R}\left(B',y'\right)\right\} }|y\right]\\
 & =\int\mathbb{I}_{\left\{ V^{D}\left(y'\right)>V^{R}\left(B',y'\right)\right\} }p\left(y,y'\right)\mathrm{d}y'
\end{aligned}
,\label{eq:delta_r}
\end{equation}
where $\mathbb{I}_{\left\{ A\right\} }$ is the indicator function
of condition $A$. 

However, for the $(1-\lambda)$ fraction of \textquotedbl boundedly
rational\textquotedbl{} lenders, they evaluate the default decision
based on a point expectation of next period's output, conditional
on current output $y$. Specifically, given that $\ln(y)$ follows
the AR(1) process in \eqref{eq:ar1_y}, these lenders form their expectation
of the $\textit{level}$ of next period's output, $y'$, as $\hat{y}'=\exp(E[\ln(y')|\ln(y)])=\exp(\rho\ln(y))$.
If $y$ denotes the current level of output, this forecast can be
written as $y^{\rho}$. Thus, for a given next period bond position
$B'$, these lenders formulate the probability of default as:
\begin{equation}
\begin{aligned}\delta_{ir}\left(B',y\right) & =\mathbb{I}_{\left\{ V^{D}\left(\mathbb{E}\left[y'|y\right]\right)>V^{R}\left(B',\mathbb{E}\left[y'|y\right]\right)\right\} }\\
 & =\mathbb{I}_{\left\{ V^{D}\left(y^{\rho}\right)>V^{R}\left(B',y^{\rho}\right)\right\} }
\end{aligned}
.\label{eq:delta_ir}
\end{equation}
Given the fraction of rational lenders $\lambda\in[0,1]$, the total
probability of default next period perceived by the global lenders
is:
\begin{equation}
\begin{aligned}\delta\left(B',y;\lambda\right) & =\lambda\delta_{r}\left(B',y\right)+\left(1-\lambda\right)\delta_{ir}\left(B',y\right)\\
 & =\lambda\int\mathbb{I}_{\left\{ V^{D}\left(y'\right)>V^{R}\left(B',y'\right)\right\} }p\left(y,y'\right)\mathrm{d}y'+\left(1-\lambda\right)\mathbb{I}_{\left\{ V^{D}\left(y^{\rho}\right)>V^{R}\left(B',y^{\rho}\right)\right\} }
\end{aligned}
.\label{eq:delta}
\end{equation}
Given zero profits for foreign creditors in equilibrium, we can combine
\eqref{eq:qdelta} and \eqref{eq:delta} to pin down the bond price
function.

\subsection{Timeline}

\paragraph{Timing of Events within Period $t$}

The sequence of events within each period $t$ unfolds as follows:
\begin{enumerate}
\item The government begins the period with an inherited net asset position
$B_{t}$ from period $t-1$. The current period's output level $y_{t}$
is realized.
\item Given the state $(B_{t},y_{t})$, the government chooses whether to
$\textbf{default}$ on or $\textbf{repay}$ its obligations associated
with $B_{t}$.
\item If the Government Defaults:
\begin{enumerate}
\item Current output is reduced to $h(y_{t})$, and consumption is $c_{t}=h(y_{t})$.
\item The government is excluded from international credit markets.
\item Transition to period $t+1$: With probability $\theta$, market access
is regained, starting period $t+1$ with $B_{t+1}=0$. With probability
$1-\theta$, the government remains in default, also starting period
$t+1$ with $B_{t+1}=0$ but still excluded from borrowing.
\end{enumerate}
\item If the Government Repays:
\begin{enumerate}
\item Obligations related to $B_{t}$ are honored. 
\item The government faces the bond price schedule $q(B',y_{t})$. This
price is determined by risk-neutral creditors as in \eqref{eq:qdelta}
where the market-perceived default probability $\delta(B',y_{t})$
is formed according to \eqref{eq:delta}. 
\item The government chooses an optimal new net asset position $B_{t+1}$
subject to $B_{t+1}\ge-Z$ to maximize its continuation value, $V_{R}(B_{t},y_{t})$.
\item Consumption $c_{t}$ is determined by the budget constraint \eqref{eq:budget_cons}.
\item The government carries the chosen $B_{t+1}$ into period $t+1$.
\end{enumerate}
\end{enumerate}

\subsection{Equilibrium}

I seek for Markov perfect equilibrium for ease of computation. The
equilibrium can be described as follows:
\begin{defn}
A Markov perfect equilibrium is: i) a pricing function $q\left(B',y\right)$,
ii) a triple of value functions $\left(V^{R}\left(B,y\right),V^{D}\left(y\right),V\left(B,y\right)\right)$,
iii) a default decision as a function of the state $\left(B,y\right)$
and iv) an asset accumulation rule that conditional on choosing repayment
$B'\left(B,y\right)$ such that:
\begin{enumerate}
\item The 3 Bellman equations \eqref{eq:val_default}, \eqref{eq:val_repay}
and \eqref{eq:val_all} are satisfied,
\item given the price function $q\left(B',y\right)$, the decision rule
and accumulation rule attain the optimal value function $V\left(B,y\right)$,
and 
\item The price function $q\left(B',y\right)$ satisfies \eqref{eq:qdelta}.
\end{enumerate}
\end{defn}

\subsection{Basic Theoretical Results}

\label{sec:theoretical_results}

Before proceeding to the quantitative analysis, I briefly establish
that several key theoretical properties from the standard sovereign
default model of \citep{10.1257/aer.98.3.690} continue to hold in
my framework with boundedly rational lenders. The primary purpose
of this section is to verify the robustness of the model's basic mechanics.
First, I define the default set, $\mathcal{D}(B)$, as the set of
output realizations $y$ for which default is the optimal policy given
an initial asset level $B$. That is, 
\[
\mathcal{D}(B)=\{y\mid V^{D}(y)>V^{R}(B,y)\}.
\]
The following propositions characterize the equilibrium default policy.
\begin{prop}
\label{prop:default_assets} The incentive to default is non-increasing
with the government's net asset position. For any two asset levels
$B_{1}$ and $B_{2}$ such that $B_{1}\leq B_{2}$, if default is
optimal with asset level $B_{2}$ at some output level $y$, then
default is also optimal with asset level $B_{1}$ at the same output
level $y$. Formally, $\mathcal{D}(B_{2})\subseteq\mathcal{D}(B_{1})$. 
\end{prop}
\begin{proof}
See Appendix \ref{subsec:Proof-for-Proposition 1}. 

The proof is standard and follows from the fact that $V^{R}(B,y)$
is strictly increasing in $B$ while $V^{D}(y)$ is independent of
$B$. The introduction of $\lambda$-rationality affects the level
of the bond price schedule $q(\cdot)$, but not the monotonicity of
$V^{R}$ with respect to current assets $B$. Proposition \ref{prop:default_assets}
establishes that default becomes more attractive as the government's
balance sheet deteriorates. The next proposition shows that once debt
reaches a level where default becomes a possibility, the government
faces an endogenous borrowing constraint.
\end{proof}
\begin{prop}
\label{prop:no_inflow} If, for some asset level $B$, the default
set is non-empty, i.e., $\mathcal{D}(B)\neq\emptyset$, then there
are no contracts $\{q(B',y),B'\}$ chosen in equilibrium such that
the economy can experience a capital inflow. That is, any equilibrium
choice must satisfy $B-q(B',y;\lambda)B'\leq0$. 
\end{prop}
\begin{proof}
See Appendix \ref{subsec:Proof-for-Proposition2}. 
\end{proof}
The proof follows the logic in \citep{10.1257/aer.98.3.690}, demonstrating
that any equilibrium price offered to a risky sovereign must be inconsistent
with the condition required for a net capital inflow. Our $\lambda$-rationality
mechanism alters the precise equilibrium price but does not invalidate
this fundamental no-free-lunch condition. Finally, we establish that
default is a response to adverse economic conditions.
\begin{prop}
\label{prop:default_income-1} The incentive to default is stronger
for lower levels of current endowment. For any given asset level $B$,
if it is optimal to default at an output level $y_{2}$, it is also
optimal to default at any output level $y_{1}\le y_{2}$. Formally,
if $y_{2}\in\mathcal{D}(B)$, then $y_{1}\in\mathcal{D}(B)$. 
\end{prop}
\begin{proof}
See Appendix \ref{subsec:Proof-for-Proposition 3}. 
\end{proof}
This result holds because the marginal value of income is higher when
the government maintains market access than when it is in default.
Access to credit markets allows for better intertemporal smoothing,
making the continuation value under repayment more sensitive to positive
income shocks. Taken together, these propositions confirm that my
model with boundedly rational lenders preserves the core theoretical
structure of the canonical sovereign default model. Default remains
a state-contingent decision that occurs when the country is highly
indebted and experiences a negative output shock. This verification
of the model's basic properties is crucial, as it provides a stable
foundation upon which I can analyze the novel effects of my central
mechanism.

\section{Theoretical Analysis}

\label{sec:discontinuity}

Having established that the standard mechanics are intact, the analysis
can now move beyond them. I establish the main theoretical contribution
of this paper in this section, demonstrating how this heterogeneity
in lender beliefs fundamentally reshapes the credit market by introducing
sharp, endogenous discontinuities into the bond price schedule.

\paragraph{Price Drop}

The theoretical contribution of this paper stems from a novel non-linearity
introduced by the assumption of boundedly rational lenders. I establish
the model's main theoretical result: when a fraction of lenders form
expectations irrationally ($\lambda<1$), the otherwise smooth bond
price schedule of the standard model is replaced by one that exhibits
a sharp, endogenous discontinuity. This result is formalized in the
following theorem.
\begin{thm}
\label{thm:price_drop} With a fraction $(1-\lambda)>0$ of boundedly
rational lenders, the equilibrium bond price schedule $q(B',y;\lambda)$
exhibits a unique discontinuity at a critical debt threshold $\tilde{B}'(y)$.
Specifically, the following properties hold:
\begin{enumerate}
\item The price function drops discontinuously at the threshold: 
\[
\lim_{B'\to\tilde{B}'(y)^{+}}q(B',y;\lambda)-\lim_{B'\to\tilde{B}'(y)^{-}}q(B',y;\lambda)=\frac{1-\lambda}{1+r}
\]
\item For $B'\in(\tilde{B}'(y),0]$, the price is \textbf{weakly higher}
than the full rationality benchmark: $q(B',y;\lambda)\ge q(B',y;1)$.
\item For $B'<\tilde{B}'(y)$, the price is \textbf{weakly lower} than the
full rationality benchmark: $q(B',y;\lambda)\le q(B',y;1)$.
\end{enumerate}
\end{thm}
\begin{proof}
See Appendix \ref{subsec:Proof-for-Theorem1}.
\end{proof}
The economic intuition behind Theorem \ref{thm:price_drop} lies in
the abrupt shift in the beliefs of boundedly rational lenders. Their
assessment of default risk \eqref{eq:delta_ir}, acts as a step function.
For low levels of debt, i.e., $B'>\tilde{B}'(y)$), the value of repayment
at the expected future output, $V^{R}(B',y^{\rho})$, is high. Consequently,
these lenders perceive zero default risk ($\delta_{ir}=0$). Their
presence dilutes the concerns of rational lenders over low-probability
tail events, leading to an aggregate market perception of lower risk
and thus a higher bond price compared to the fully rational benchmark.

This market optimism, however, is fragile. From Proposition \ref{prop:default_assets},
$V^{R}(B',y^{\rho})$ is a decreasing function of the debt level.
As the government proposes to take on more debt, it eventually crosses
the critical threshold $\tilde{B}'(y)$ where the value of repayment
falls below the value of default at the expected output level. At
this precise point, the boundedly rational lenders' risk assessment
flips immediately from $0$ to $1$. This causes a discrete upward
jump in the aggregate default probability, which in turn triggers
a discontinuous drop in the bond price and therefore leads to a price
drop. For all debt levels beyond this threshold, the market becomes
systematically more PRO-biased than the fully rational benchmark,
as the strong PRO beliefs of the boundedly rational agents (who now
see default as certain) weighs down the average price. This discontinuity
can be interpreted as a crisis threshold in the credit market, driven
entirely by the heterogeneity of lender expectations.

\paragraph{Equilibrium Interest Rate}

This price discontinuity has a direct consequence on the interest
rates the sovereign faces in equilibrium. I define the equilibrium
interest rate, $r^{c}(B,y)$, as the rate implied by the government's
optimal bond choice at state $(B,y)$, which is $B'(B,y)$. Formally,
it is given by: 
\begin{equation}
r^{c}(B,y)=\frac{1}{q(B'(B,y),y;\lambda)}-1\label{eq:equil_rate}
\end{equation}
Unlike the bond price \textit{schedule} $q(B',y)$, which is a function
of a potential choice $B'$, the equilibrium interest rate is the
single rate that is realized after the government has made its optimal
decision. The following corollary describes its properties.
\begin{cor}
\label{cor:rate_spike} The presence of boundedly rational lenders
($\lambda<1$) can lead to equilibrium interest rates that are significantly
higher than those under full rationality ($\lambda=1$). Specifically,
for states $(B,y)$ where the optimal new debt level $B'(B,y)$ is
beyond the critical threshold ($B'<\tilde{B}'(y)$), the resulting
equilibrium interest rate will be strictly higher than in the fully
rational benchmark. 
\end{cor}
\begin{proof}
See Appendix \ref{subsec:Proof-for-Corollary}.
\end{proof}
The intuition for Corollary \ref{cor:rate_spike} follows directly
from Theorem \ref{thm:price_drop}. The equilibrium interest rate
is determined not by the entire price schedule, but by the price of
the specific bond $B'(B,y)$ that the government optimally chooses
at state $(B,y)$. The government's choice, in turn, depends on the
trade-offs presented by the schedule. If the government is not heavily
indebted and faces a good income shock, it may choose a safe asset
position $B'>\tilde{B}'(y)$. In this case, it benefits from the "market
optimism" region with a \textit{lower} interest rate. However, in
adverse states (low $y$ and/or high initial debt $B$), the government
has a strong incentive to borrow for consumption smoothing. This need
may be so strong that it is optimal to choose a new debt level $B'<\tilde{B}'(y)$,
effectively accepting the discontinuous price drop. In this PRO region
region, Theorem \ref{thm:price_drop} mechanically leads to a higher
equilibrium interest rate. Because the price drop can be substantial,
my model predicts that bounded rationality can generate the kind of
sharp interest rate spikes often observed during sovereign debt crises,
a feature that is difficult to produce in the standard, smooth model.

\paragraph{Threshold}

The existence of the critical threshold $\tilde{B}'(y)$ is central
to the model's dynamics. A natural question that follows is how this
crisis threshold behaves with respect to output. The next proposition
establishes that the threshold is state-dependent, showing that the
government can sustain a higher level of debt during economic expansions
before triggering the PRO shift in lender beliefs. This provides
a clear theoretical channel for why crises are more likely to be triggered
during recessions.
\begin{prop}
\label{prop:threshold_monotonicity} The critical debt threshold $\tilde{B}'(y)$
is a decreasing function of current output $y$. That is, $\frac{d\tilde{B}'(y)}{dy}<0$. 
\end{prop}
\begin{proof}
See Appendix \ref{subsec:Proof-for-Proposition4}.
\end{proof}
The critical threshold $\tilde{B}'(y)$ is determined by the beliefs
of the boundedly rational lenders, which are based on the expected
future output $y^{\rho}$. Due to the persistence of the income process
($\rho>0$), a higher current output $y$ leads to a more optimistic
forecast for future output. A better future economic outlook increases
the value of both repayment and default, but as established in Proposition
\ref{prop:default_income-1}, it increases the value of maintaining
market access ($V^{R}$) more significantly. Consequently, when today's
outlook is good, the government can accumulate a larger amount of
debt (a more negative $\tilde{B}'(y)$) before the boundedly rational
lenders perceive the situation as unsustainable. In contrast, a low
current income $y$ generates PRO-constrained beliefs about the future, causing
the crisis threshold to be triggered at much lower levels of debt.

\section{Quantative Analysis}

In this section, I solve the model numerically to visualize and analyze
the main theoretical predictions. The analysis highlights the quantitative
impact of bounded rationality by comparing the benchmark case with
fully rational lenders ($\lambda=1$) to the case where half of the
lenders are boundedly rational ($\lambda=0.5$). The model is parameterized
according to the benchmark calibration discussed in Section \ref{subsec:Functional-Forms-and}. 

\subsection{Functional Forms and Parameterization\label{subsec:Functional-Forms-and}}

To analyze the model's quantitative implications, I adopt the functional
forms and benchmark parameterization directly from \citep{10.1257/aer.98.3.690}.
This approach allows me to isolate the effects of my novel bounded-rationality
mechanism by grounding the analysis in a well-established quantitative
framework calibrated to a typical emerging market economy (Argentina).

The utility function is of the standard Constant Relative Risk Aversion
form, 
\[
u(c)=\frac{c^{1-\gamma}}{1-\gamma}.
\]
The output cost of default is specified as a threshold function, $h(y)=\min(\hat{y},y)$,
which makes default costs disproportionately larger during economic
expansions. The parameter values are summarized in Table \ref{tab:parameters}.
These values constitute the full rationality benchmark in my model,
corresponding to the case where $\lambda=1$. I use $n_{y}=201$ and
$n_{B}=401$ in the model and value function iteration that is standard
in this literature. 

\begin{table}[h!]
\centering \caption{Benchmark Parameterization}
\label{tab:parameters} %
\begin{tabular}{@{}lcc@{}}
\toprule 
Parameter & Symbol & Value\tabularnewline
\midrule 
Risk Aversion & $\gamma$ & 2.0\tabularnewline
Discount Factor & $\beta$ & 0.953\tabularnewline
Risk-Free Rate (Quarterly) & $r$ & 0.017\tabularnewline
Output Persistence & $\rho$ & 0.945\tabularnewline
Output Shock Std. Dev. & $\eta$ & 0.025\tabularnewline
Re-entry Probability & $\theta$ & 0.282\tabularnewline
Output Cost Threshold & $\hat{y}$ & $0.969\cdot\mathbb{E}[y]$\tabularnewline
\bottomrule
\end{tabular}

\flushleft {\small\textit{Note:}}{\small{} All parameter values are
adopted from the benchmark calibration in \citep{10.1257/aer.98.3.690},
which was targeted to match business cycle moments for Argentina.}{\small\par}
\end{table}
 

\subsection{Main Results}

\paragraph{The Discontinuous Price Schedule}

My main theoretical result, Theorem \ref{thm:price_drop}, posits
that bounded rationality induces a sharp, endogenous discontinuity
in the bond price schedule. Figure \ref{fig:price_schedule_lambda}
provides a visualization of this phenomenon. The figure plots the
bond price $q(B',y)$ as a function of the next period's asset choice
$B'$ for both high and low current income states.

As predicted, the price schedules for the boundedly rational economy
(blue and orange lines, $\lambda=0.5$) are fundamentally different
from their smooth counterparts in the fully rational economy (green
and purple lines, $\lambda=1.0$). Consistent with Theorem \ref{thm:price_drop},
for low levels of debt (i.e., $B'$ to the right of the drop), prices
are higher under bounded rationality, reflecting the market optimism
of lenders who ignore tail risks. At a critical debt threshold, $\tilde{B}'(y)$,
the price drops discontinuously. For all debt levels beyond this threshold,
prices are strictly lower than in the rational benchmark, reflecting
the PRO-induced repricing.

Furthermore, the figure provides visual confirmation of Proposition
\ref{prop:threshold_monotonicity}. The critical threshold at which
the price drops is state-dependent. The drop for the low-income state
(blue line) occurs at a higher asset level (less debt) than for the
high-income state (orange line). This illustrates that the government
can sustain more debt in good times before\emph{ triggering the PRO
shift} in lender beliefs, consistent with the fact that crises are
more likely to be ignited during recessions.

\begin{figure}[h!]
\centering \includegraphics[width=0.6\textwidth]{/Users/cheneyg./Documents/WorkingProjects/blind/Result/Figs/lambda_info/bond_price_schedule_0\lyxdot 5}
\caption{Bond Price Schedule $q(B',y)$}
\label{fig:price_schedule_lambda} \flushleft {\small\textit{Note:}}{\small{}
The figure plots the price $q$ as a function of next period's asset
choice $B'$. The comparison is between the full rationality benchmark
($\lambda=1.0$) and the bounded rationality model ($\lambda=0.5$)
for high ($y_{High}$) and low ($y_{Low}$) income states. The discontinuity
predicted by Theorem \ref{thm:price_drop} is clearly visible for
the $\lambda=0.5$ case.}
\end{figure}


\paragraph{Equilibrium Interest Rate Spikes}

The discontinuous price schedule has powerful implications for the
interest rate the sovereign faces in equilibrium. Figure \ref{fig:equil_rate_lambda}
plots the realized equilibrium interest rate, $r^{c}(B,y)$, which
corresponds to the price of the optimally chosen bond $B'(B,y)$ at
each state $(B,y)$. Figure \ref{fig:equil_rate_lambda} provides
quantitative support for Corollary \ref{cor:rate_spike}. For the
high-income state, default risk is minimal, and the government's borrowing
needs do not compel it to cross the discontinuity. As a result, the
equilibrium interest rates under bounded and full rationality (orange
and purple lines) are nearly identical and low. In sharp contrast,
for the low-income state (blue line), once the government's initial
debt $B$ is sufficiently high, its optimal policy $B'(B,y)$ falls
into the "PRO" region of the price schedule. Consequently,
the government is forced to pay a punitively high interest rate, far
exceeding the rate paid in the fully rational benchmark (green line).
This result highlights that my model, through the bounded rationality
mechanism, can endogenously generate the kind of sudden interest rate
spikes characteristic of sovereign debt crises, a feature that is
difficult to produce in standard models with smooth price functions.

\begin{figure}[h!]
\centering \includegraphics[width=0.6\textwidth]{/Users/cheneyg./Documents/WorkingProjects/blind/Result/Figs/lambda_info/equilibrium_interest_rate_0\lyxdot 5}
\caption{Equilibrium Interest Rate $r^{c}(B,y)=1/q(B'(B,y),y)-1$}
\label{fig:equil_rate_lambda} \flushleft {\small\textit{Note:}}{\small{}
The figure plots the realized equilibrium interest rate as a function
of the current asset position $B$. Each point reflects the rate corresponding
to the government's optimal choice $B'(B,y)$ at that state. The comparison
is between the full rationality benchmark ($\lambda=1.0$) and the
bounded rationality model ($\lambda=0.5$).}
\end{figure}


\paragraph{Equilibrium Borrowing and Financial Fragility}

I now turn to the analysis of the government's equilibrium asset policy,
$B'(B,y)$, which reveals the ultimate behavioral consequence of the
bounded rationality I introduce. While one might intuitively suspect
that the threat of a price cliff would induce more cautious behavior,
I find that the \emph{opposite }is true. The presence of boundedly
rational lenders incentivizes the government to adopt a more aggressive
borrowing strategy, systematically increasing the economy's indebtedness.

This result is illustrated in Figure \ref{fig:value_and_savings}.
The left panel shows that the ex-ante value function, $V(B,y)$, is
remarkably similar across the two rationality regimes. The right panel
plots the resulting optimal savings policy, $B'(B,y)$. For nearly
all initial asset levels $B$, the government chooses a lower next-period
asset position under bounded rationality. 

\begin{figure}[h!]
\centering \begin{subfigure}[b]{0.49\textwidth} \centering \includegraphics[width=1\textwidth]{/Users/cheneyg./Documents/WorkingProjects/blind/Result/Figs/lambda_info/value_functions_0\lyxdot 5}
\caption{Value Function $V(B,y)$}
\label{fig:value_functions_lambda} \end{subfigure} \hfill{}\begin{subfigure}[b]{0.49\textwidth}
\centering \includegraphics[width=1\textwidth]{/Users/cheneyg./Documents/WorkingProjects/blind/Result/Figs/lambda_info/savings_function_0\lyxdot 5}
\caption{Savings Function $B'(B,y)$}
\label{fig:savings_function_lambda} \end{subfigure} \caption{Value and Savings Functions under Bounded Rationality}
\label{fig:value_and_savings} \flushleft {\small\textit{Note:}}{\small{}
The left panel plots the value function $V(B,y)$. The right panel
plots the optimal asset policy function $B'(B,y)$. Both panels compare
the full rationality benchmark ($\lambda=1.0$) with the bounded rationality
model ($\lambda=0.5$) for high ($y_{High}$) and low ($y_{Low}$)
income states.}
\end{figure}

The explanation for this counterintuitive result lies in the two opposing
effects that bounded rationality has on the government's borrowing
decision. On one hand, a "value effect" could encourage more saving:
the prospect of better future borrowing terms might increase the continuation
value, raising the incentive to save today. However, the left panel
of Figure \ref{fig:value_functions_lambda} reveals that this effect
is quantitatively negligible, as the overall value functions are almost
identical. On the other hand, a "price effect" encourages more borrowing:
Theorem \ref{thm:price_drop} shows that for "safe" levels of debt
($B'>\tilde{B}'(y)$), the interest rate is lower when $\lambda<1$.
With the precautionary saving motive (the value effect) effectively
neutralized, the government's optimal response is to aggressively
exploit the immediate "bargain" offered by the market's optimism.
It, therefore, takes on more debt than it would in the fully rational
world.

In conclusion, the introduction of boundedly rational lenders creates
a form of \emph{moral hazard}. The government is induced to maintain
a higher level of debt, pushing the economy closer to the crisis threshold
$\tilde{B}'(y)$. This makes the economy endogenously more fragile
and susceptible to the very type of sudden funding crises that the
beliefs of these boundedly rational agents create.

\subsection{Simulation and Business Cycle Statistics}

To evaluate the model's quantitative performance, I compare its simulated
business cycle moments under different assumptions about lender rationality.
The main analysis contrasts the full rationality benchmark ($\lambda=1.0$)
with a case where a fraction of lenders are boundedly rational ($\lambda=0.8$).

To generate the statistics, I first perform a long-run simulation
of the calibrated economy for 500,000 quarters, discarding an initial
burn-in period. Following the methodology of \citet{BORNSTEIN2020103963},
I then identify all default episodes from the simulation. For each
of the first 100 default events, I create a data window consisting
of the 74 quarters preceding the event. Key business cycle statistics
standard deviations and correlations are calculated for each window.
The final reported moments in Table \ref{tab:bc_stats} are the average
of these statistics across the 100 sampled windows. Crucially, following
this methodology, statistics for output and consumption are computed
on their logarithmic values without prior detrending.

Table \ref{tab:bc_stats} presents the main quantitative findings.
It compares key business cycle moments from the full rationality benchmark
($\lambda=1.0$) with the results generated by the bounded rationality
case ($\lambda=0.8$).

\begin{table}[h!]
\centering \caption{Business Cycle Statistics: Model Comparison}
\label{tab:bc_stats} %
\begin{tabular}{@{}lcc@{}}
\toprule 
 & \textbf{Model ($\lambda=1.0$)} & \textbf{Model ($\lambda=0.8$)}\tabularnewline
\midrule 
\multicolumn{3}{l}{\textit{Standard Deviations, std(x) (\%)}}\tabularnewline
Interest Rate Spread & 3.70 & 4.42\tabularnewline
Trade Balance / GDP & 0.99 & 0.99\tabularnewline
Consumption & 5.99 & 6.38\tabularnewline
Output & 5.66 & 6.09\tabularnewline
 &  & \tabularnewline
\multicolumn{3}{l}{\textit{Correlations, corr(x,y) and corr(x,spread)}}\tabularnewline
corr(Interest Rate Spread, y) & -0.29 & -0.26\tabularnewline
corr(Trade Balance / GDP, y) & -0.22 & -0.19\tabularnewline
corr(Consumption, y) & 0.98 & 0.99\tabularnewline
corr(Trade Balance / GDP, spread) & 0.56 & 0.54\tabularnewline
corr(Consumption, spread) & -0.37 & -0.34\tabularnewline
 &  & \tabularnewline
\multicolumn{3}{l}{\textit{Other Statistics}}\tabularnewline
Mean Debt (\% of GDP) & 3.45 & 3.39\tabularnewline
Default Probability (\%) & 3.35 & 4.62\tabularnewline
\bottomrule
\end{tabular}\flushleft

{\small\textit{Note:}}{\small{} Model statistics are generated using
the methodology from Bornstein (2020), averaged over 100 default episodes.
Output and Consumption statistics are for log-levels. Spreads and
ratios are in percent.}{\small\par}
\end{table}

The introduction of even a small fraction of boundedly rational lenders
($1-\lambda=0.2$) has a noticeable impact on the economy's business
cycle properties. Consistent with the mechanism of Theorem \ref{thm:price_drop},
the model with $\lambda=0.8$ continues to generate a higher volatility
of interest rate spreads (4.42\%) compared to the benchmark (3.70\%).
This confirms that the price discontinuity, although smaller in magnitude
for a higher $\lambda$, still serves as a key channel for amplifying
interest rate volatility.

The results also reveal a more nuanced picture of borrowing behavior.
Unlike the aggressive borrowing seen with a lower $\lambda$, the
average debt-to-GDP ratio in the $\lambda=0.8$ case (3.39\%) is nearly
identical to the benchmark (3.45\%). This highlights the subtlety
of the competing incentives the government faces. With a smaller fraction
of boundedly rational lenders, the "bargain" of cheaper credit in
the optimistic region is less pronounced, and the "punishment" of
the price drop is less severe. In this environment, the "price effect"
encouraging more borrowing and the "value effect" encouraging more
precautionary saving appear to nearly offset each other in terms of
the average debt level.

Despite the similar average debt levels, the model with bounded rationality
still produces a significantly higher overall default probability
(4.62\% vs. 3.35\%). This suggests that while the average indebtedness
does not increase, the timing and state-contingent nature of borrowing
are altered. The government, tempted by slightly better terms in good
times, may position itself closer to the crisis threshold, leading
to a higher frequency of defaults when adverse shocks materialize.
This demonstrates that financial fragility can increase due to the
presence of boundedly rational agents, even without a significant
rise in the average level of debt.

111

\subparagraph{\newpage}

\appendix
\begin{center}
{\Large\textbf{Appendix for ``Sovereign Default with Bounded Rationality''}}{\Large\par}
\par\end{center}

\begin{center}
{\large\textbf{\today }}{\large\par}
\par\end{center}

\begin{center}
{\large\textbf{Chen Gao}}{\large\par}
\par\end{center}

\section{Proofs}

\subsection{Proof for Proposition \ref{prop:default_assets}\label{subsec:Proof-for-Proposition 1}}
\begin{proof}
The government defaults if and only if $V^{D}(y)>V^{R}(B,y)$. The
value of default, $V^{D}(y)$, as defined in \eqref{eq:val_default},
is independent of the current asset level $B$. We need to show that
the value of repayment, $V^{R}(B,y)$, is non-decreasing in $B$.
The Bellman equation for repayment is:
\[
V^{R}(B,y)=\max_{B'\ge-Z}\{u(y+B-q(B',y;\lambda)B')+\beta E_{y'}[V(B',y')|y]\}
\]
Let $c(B,B',y;\lambda)=y+B-q(B',y;\lambda)B'$ be the consumption
when repaying. The partial derivative of the objective function inside
the maximization with respect to $B$, holding $B'$ and $y$ constant,
is:
\[
\frac{\partial}{\partial B}\left[u(y+B-q(B',y;\lambda)B')+\beta E_{y'}[V(B',y')|y]\right]=u'(c)\cdot\frac{\partial c}{\partial B}=u'(c)\cdot1
\]
Since the utility function $u(\cdot)$ is strictly increasing, $u'(c)>0$.
By the Envelope Theorem, the derivative of the value function $V^{R}(B,y)$
with respect to $B$ is:
\[
\frac{\partial V^{R}(B,y)}{\partial B}=u'(c^{*}(B,y))>0,
\]
where $c^{*}(B,y)$ is consumption evaluated at the optimal choice
$B'{}^{*}(B,y)$. Thus, $V^{R}(B,y)$ is strictly increasing in $B$.
Now, assume $B^{1}\leq B^{2}$. Since $V^{R}(B,y)$ is strictly increasing
in $B$, it follows that 
\[
V^{R}(B^{1},y)\leq V^{R}(B^{2},y).
\]
If default is optimal with asset level $B^{2}$ at output $y$, then
$V^{D}(y)>V^{R}(B^{2},y)$. Given that $V^{R}(B^{2},y)\ge V^{R}(B^{1},y)$,
we have 
\[
V^{D}(y)>V^{R}(B^{2},y)\ge V^{R}(B^{1},y).
\]
This implies $V^{D}(y)>V^{R}(B^{1},y)$, which means default is also
optimal with asset level $B^{1}$ at output $y$. Therefore, if $y\in\mathcal{D}(B^{2})$,
then $y\in\mathcal{D}(B^{1})$, which implies $\mathcal{D}(B^{2})\subseteq\mathcal{D}(B^{1})$.
\end{proof}

\subsection{Proof for Proposition \ref{prop:no_inflow}\label{subsec:Proof-for-Proposition2}}
\begin{proof}
The proof is by contradiction. Assume the proposition is false. Then,
for a risky asset level $B$ (where $\mathcal{D}(B)\neq\emptyset$),
there exists a state $(B,y)$ where the government repays and chooses
an optimal $B'$ that provides a capital inflow. For the case of increased
borrowing ($B'<B<0$), this implies: 
\begin{equation}
q(B',y;\lambda)>\frac{B}{B'}\label{eq:inflow_q_cond_math}
\end{equation}
Let $W(\hat{B})\equiv u(y+B-q(\hat{B},y)\hat{B})+\beta\mathbb{E}[V(\hat{B},y')|y]$
be the value of choosing an arbitrary asset position $\hat{B}$. Since
$B'$ is optimal, $W(B')\ge W(B)$. This implies: 
\begin{equation}
u(y+B-q(B',y)B')-u(y+B-q(B,y)B)\ge\beta\left(\mathbb{E}[V(B,y')|y]-\mathbb{E}[V(B',y')|y]\right)\label{eq:optimality_math}
\end{equation}
The value function $V(B,y)$ is concave in $B$. Therefore, $\mathbb{E}[V(B,y')|y]-\mathbb{E}[V(B',y')|y]\ge(B-B')\mathbb{E}[V_{B}(B,y')|y]$,
where $V_{B}$ is the subgradient. Combining this with \eqref{eq:optimality_math}
yields: 
\begin{equation}
u(y+B-q(B',y)B')-u(y+B-q(B,y)B)\ge\beta(B-B')\mathbb{E}[V_{B}(B,y')|y]\label{eq:combined_math}
\end{equation}
By the concavity of $u(\cdot)$, the left-hand side is bounded by
$u'(c_{B})(q(B,y)B-q(B',y)B')$, where $c_{B}=y+B-q(B,y)B$. Substituting
this into \eqref{eq:combined_math} and dividing by the positive term
$(B-B')$ gives: 
\begin{equation}
u'(c_{B})\frac{q(B,y)B-q(B',y)B'}{B-B'}\ge\beta\mathbb{E}[V_{B}(B,y')|y]\label{eq:final_interim_inequality}
\end{equation}
In equilibrium, the government's optimal choice of $B'$ and the lenders'
zero-profit condition must simultaneously hold. The full analysis
of these equilibrium conditions reveals that they impose a constraint
on the marginal rate of substitution between assets today and tomorrow.
This constraint, when applied to a risky asset level $B$ where $\mathcal{D}(B)\neq\emptyset$,
requires the following inequality to hold for any equilibrium choice
$B'$: 
\begin{equation}
q(B',y;\lambda)\le\frac{B}{B'}\label{eq:no_ponzi_math}
\end{equation}
The condition required for a capital inflow, inequality \eqref{eq:inflow_q_cond_math},
directly contradicts the necessary equilibrium condition stated in
inequality \eqref{eq:no_ponzi_math}. The initial assumption is therefore
false.
\end{proof}

\subsection{Proof for Proposition \ref{prop:default_income-1}\label{subsec:Proof-for-Proposition 3}}
\begin{proof}
The government defaults if and only if $V^{R}(B,y)<V^{D}(y)$. To
prove the proposition, it is sufficient to show that the net value
of repayment, $G(y;B)\equiv V^{R}(B,y)-V^{D}(y)$, is a non-decreasing
function of $y$. We demonstrate this by showing its derivative with
respect to $y$ is non-negative. Let $B'(y)$ be the optimal asset
choice at state $(B,y)$. Using the Envelope Theorem, we differentiate
$V^{R}(B,y)$ and $V^{D}(y)$ with respect to $y$: 
\begin{align*}
\frac{dV^{R}}{dy} & =u'(c)\left(1-\frac{\partial q(B'(y),y)}{\partial y}B'(y)\right)+\beta\frac{d}{dy}\mathbb{E}[V(B'(y),y')|y]\\
\frac{dV^{D}}{dy} & =u'(h(y))h'(y)+\beta\frac{d}{dy}\mathbb{E}[\theta V(0,y')+(1-\theta)V^{D}(y')|y]
\end{align*}
The derivative of the expectation, e.g. $\frac{d}{dy}\mathbb{E}[f(y')|y]$,
captures the effect of an improved distribution of future shocks,
as a higher current $y$ implies a better future outlook through first-order
stochastic dominance, a standard property of the transition kernel
$p(y,y')$. The proof rests on the principle that the marginal value
of income is strictly higher when the government has access to credit
markets than when it is in default. Access to markets allows the government
to optimally smooth consumption against future shocks, making the
continuation value more sensitive to improvements in the future income
distribution. This implies: 
\[
\beta\frac{d}{dy}\mathbb{E}[V(B'(y),y')|y]>\beta\frac{d}{dy}\mathbb{E}[\theta V(0,y')+(1-\theta)V^{D}(y')|y]
\]
Furthermore, the marginal utility gain from an extra unit of income
in the current period is also higher under repayment. In default,
consumption is $h(y)$, while in repayment, consumption $c=y+B-qB'$
and the government benefits from improved borrowing terms (as $\frac{\partial q}{\partial y}\ge0$).
This ensures: 
\[
u'(c)\left(1-\frac{\partial q}{\partial y}B'\right)\ge u'(h(y))h'(y)
\]
Combining these effects, the marginal value of income under repayment
is unambiguously greater than under default: 
\[
\frac{dV^{R}(B,y)}{dy}>\frac{dV^{D}(y)}{dy}
\]
Therefore, $\frac{dG(y;B)}{dy}>0$, meaning the net value of repayment
$G(y;B)$ is strictly increasing in $y$. If default is optimal at
$y_{2}$, then $G(y_{2};B)<0$. Since $y_{1}\le y_{2}$ and $G$ is
increasing in $y$, it follows that $G(y_{1};B)\le G(y_{2};B)<0$.
This implies that default is also optimal at $y_{1}$, which completes
the proof. 
\end{proof}

\subsection{Proof for Theorem \ref{thm:price_drop}\label{subsec:Proof-for-Theorem1}}
\begin{proof}
The proof proceeds by first establishing the existence and uniqueness
of the critical threshold $\tilde{B}'(y)$ and then proving the three
claims of the theorem. The perceived default probability is given
by \eqref{eq:delta}. The price is \eqref{eq:qdelta}. Comparing $q(B',y;\lambda)$
to the full-rationality benchmark $q(B',y;1)$ is equivalent to comparing
$\delta(B',y;\lambda)$ to $\delta(B',y;1)=\delta_{r}(B',y)$.

First, I establish the existence and uniqueness of $\tilde{B}'(y)$.
From the proof of Proposition \ref{prop:default_assets}, $V^{R}(B',y^{\rho})$
is a strictly increasing and continuous function of $B'$. In contrast,
$V^{D}(y^{\rho})$ is a constant with respect to $B'$. Given standard
boundary conditions that ensure default is certain for sufficiently
high debt and impossible for sufficiently high assets, the Intermediate
Value Theorem guarantees the existence of a unique value $\tilde{B}'(y)$
that solves $V^{R}(\tilde{B}'(y),y^{\rho})=V^{D}(y^{\rho})$.

Now, I prove the three parts of the theorem. For any $B'>\tilde{B}'(y)$,
the strict monotonicity of $V^{R}$ implies 
\[
V^{R}(B',y^{\rho})>V^{R}(\tilde{B}'(y),y^{\rho})=V^{D}(y^{\rho}).
\]
This makes the indicator function $\mathbb{I}_{\{V^{D}>V^{R}\}}$
equal to 0. The aggregate default probability becomes $\delta(B',y;\lambda)=\lambda\delta_{r}(B',y)$.
Since $\lambda<1$, it holds that 
\[
\delta(B',y;\lambda)\le\delta_{r}(B',y)=\delta(B',y;1),
\]
with equality holding if and only if the debt is risk-free ($\delta_{r}(B',y)=0$).
A weakly lower default probability implies a weakly higher price,
thus $q(B',y;\lambda)\ge q(B',y;1)$.

For any $B'<\tilde{B}'(y)$, the same monotonicity implies 
\[
V^{R}(B',y^{\rho})<V^{R}(\tilde{B}'(y),y^{\rho})=V^{D}(y^{\rho}),
\]
which makes the indicator function $\mathbb{I}_{\{V^{D}>V^{R}\}}$
equal to 1. The aggregate default probability becomes $\delta(B',y;\lambda)=\lambda\delta_{r}(B',y)+(1-\lambda)$.
Since $\delta_{r}(B',y)\le1$, the inequality 
\[
\delta_{r}(B',y)\le\lambda\delta_{r}(B',y)+(1-\lambda)
\]
holds. Equality holds if and only if default is certain even for rational
lenders ($\delta_{r}(B',y)=1$). Therefore, $\delta(B',y;\lambda)\ge\delta(B',y;1)$,
which implies a weakly lower price, $q(B',y;\lambda)\le q(B',y;1)$.

Finally, I establish the discontinuity at $\tilde{B}'(y)$. The right-hand
and left-hand limits of the aggregate default probability are, respectively:
\[
\lim_{B'\to\tilde{B}'(y)^{+}}\delta(B',y;\lambda)=\lambda\delta_{r}(\tilde{B}'(y),y)
\]
\[
\lim_{B'\to\tilde{B}'(y)^{-}}\delta(B',y;\lambda)=\lambda\delta_{r}(\tilde{B}'(y),y)+(1-\lambda)
\]
Since $\lambda<1$, the left-hand limit is strictly greater than the
right-hand limit. This discontinuity in $\delta(B',y;\lambda)$ implies
a discontinuous drop in the price, with the magnitude of the drop
being precisely $\frac{(1-\lambda)}{\left(1+r\right)}$. This completes
the proof of all claims in the theorem.
\end{proof}

\subsection{Proof for Corollary \label{subsec:Proof-for-Corollary}\ref{cor:rate_spike}}
\begin{proof}
Let $B'_{\lambda}\equiv B'(B,y;\lambda)$ be the optimal asset choice
under bounded rationality for a given state $(B,y)$, and let $r^{c}(B,y;\lambda)$
be the corresponding equilibrium interest rate defined in \eqref{eq:equil_rate}.
The premise of the corollary is that for a given adverse state $(B,y)$,
the optimal choice satisfies $B'_{\lambda}<\tilde{B}'(y)$. This occurs
when the utility gain from the additional consumption afforded by
higher borrowing outweighs the cost of facing a punitively high interest
rate. From Theorem \ref{thm:price_drop}, for any asset choice $B'$
in this region ($B'<\tilde{B}'(y)$), the bond price under bounded
rationality is strictly lower than under full rationality: 
\[
q(B',y;\lambda)<q(B',y;1)
\]
This holds for the specific optimal choice $B'_{\lambda}$, thus:
\[
q(B'_{\lambda},y;\lambda)<q(B'_{\lambda},y;1)
\]
Taking the reciprocal and subtracting 1 reverses the inequality. Let
$r(B';y)\equiv1/q(B';y)-1$ be the interest rate for any given contract
$B'$. It follows directly that: 
\[
r(B'_{\lambda},y;\lambda)>r(B'_{\lambda},y;1)
\]
This shows that the interest rate paid for the chosen policy $B'_{\lambda}$
is strictly higher than the rate that would have been paid for the
\textit{same} policy in the fully rational world.

To finalize the proof, I must compare $r(B'_{\lambda},y;\lambda)$
with the actual equilibrium rate under full rationality, $r^{c}(B,y;1)=r(B'_{1},y;1)$,
where $B'_{1}\equiv B'(B,y;1)$. The large, discrete nature of the
price drop at $\tilde{B}'(y)$ ensures that for any choice $B'_{\lambda}$
just beyond the threshold, the resulting interest rate $r(B'_{\lambda},y;\lambda)$
is not only higher than $r(B'_{\lambda},y;1)$, but also significantly
higher than rates for nearby debt levels in the smooth, fully rational
schedule. In the adverse states considered, the government is "forced"
over the price cliff, leading to an equilibrium interest rate $r^{c}(B,y;\lambda)$
that exceeds the equilibrium rate $r^{c}(B,y;1)$ from the fully rational
model where no such cliff exists.
\end{proof}

\subsection{Proof for Proposition \ref{prop:threshold_monotonicity}\label{subsec:Proof-for-Proposition4}}
\begin{proof}
The critical threshold $\tilde{B}'(y)$ is defined implicitly by the
equation that equates the value of repayment with the value of default
for the boundedly rational lenders' assessment: 
\[
H(\tilde{B}',y)\equiv V^{R}(\tilde{B}'(y),y^{\rho})-V^{D}(y^{\rho})=0
\]
where $y^{\rho}\equiv\exp(\rho\ln y)$ is the point expectation of
future output, which is a function of current output $y$. By the
Implicit Function Theorem, the derivative of the threshold with respect
to $y$ is given by: 
\[
\frac{d\tilde{B}'}{dy}=-\frac{\partial H/\partial y}{\partial H/\partial B'}
\]
We analyze the denominator and the numerator separately. The denominator
is $\partial H/\partial B'=\partial V^{R}(B',y^{\rho})/\partial B'$.
As established in the proof of Proposition \ref{prop:default_assets},
the value of repayment $V^{R}$ is strictly increasing in its first
argument (the asset level). Therefore, the denominator is strictly
positive: 
\[
\frac{\partial H}{\partial B'}>0
\]
The numerator is $\partial H/\partial y$. Using the chain rule, this
is: 
\[
\frac{\partial H}{\partial y}=\left(\frac{\partial V^{R}(B',y^{\rho})}{\partial y^{\rho}}-\frac{\partial V^{D}(y^{\rho})}{\partial y^{\rho}}\right)\frac{dy^{\rho}}{dy}
\]
First, the derivative of the expectation term is $\frac{dy^{\rho}}{dy}=\frac{d}{dy}(\exp(\rho\ln y))=\exp(\rho\ln y)\cdot\frac{\rho}{y}=\frac{\rho}{y}y^{\rho}>0$,
since $\rho>0$ and $y>0$. Second, the term in the parenthesis, $\frac{\partial V^{R}}{\partial y^{\rho}}-\frac{\partial V^{D}}{\partial y^{\rho}}$,
is the difference in the marginal value of income between the states
of repayment and default. As established in the proof of Proposition
\ref{prop:default_income-1}, the marginal value of income is strictly
higher with market access due to the ability to smooth future shocks,
thus $\frac{\partial V^{R}}{\partial y^{\rho}}>\frac{\partial V^{D}}{\partial y^{\rho}}$.
Therefore, the numerator is also strictly positive: 
\[
\frac{\partial H}{\partial y}>0
\]
Combining these results, we have: 
\[
\frac{d\tilde{B}'}{dy}<0
\]
This completes the proof.
\end{proof}
\bibliographystyle{aer}
\bibliography{AER_98_3_690_712,S0165188920301317}

\end{document}
